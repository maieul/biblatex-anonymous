\documentclass{ltxdockit}[2011/03/25]
\usepackage{btxdockit}
\usepackage{fontspec}
\usepackage[mono=false]{libertine}
\usepackage{microtype}
\usepackage[main=english,french,latin]{babel}
\usepackage[strict]{csquotes}
\setmonofont[Scale=MatchLowercase]{DejaVu Sans Mono}
\usepackage{shortvrb}
\usepackage{pifont}
\usepackage{minted}
% Usefull commands
\newcommand{\biblatex}{biblatex\xspace}
\pretocmd{\bibfield}{\sloppy}{}{}
\pretocmd{\bibtype}{\sloppy}{}{}
\newcommand{\namebibstyle}[1]{\texttt{#1}}
% Meta-datas
\titlepage{%
	title={Managing anonymous work with biblatex},
	subtitle={},
	email={maieul <at> maieul <dot> net},
	author={Maïeul Rouquette},
	revision={2.3.0},
	date={08/09/2016},
	url={https://github.com/maieul/biblatex-anonymous}}

\begin{document}


\printtitlepage
\tableofcontents

\section{Introduction}

In some scholar fields, like history or philology, we have to work with anonymous work. 

This package provides facility to manage them with \biblatex, with citation and bibliography style of the \namebibstyle{verbose} family.

\section{What does the package do?}

This package does those things for  \bibtype{book}, \bibtype{bookinbook}, \bibtype{bookinarticle}, \bibtype{bookinincollection}, \bibtype{mvbook}, \bibtype{inbook}:
\begin{enumerate}
	

	\item Changing label name to author only, never editor.
	\item Providing a sorting scheme to put all anonymous work at the begining of a bibliography. This sorting scheme is derivated from \verb+nyt+ scheme.
	\item For French, changing \enquote{\selectlanguage{french}dir.} by \enquote{\selectlanguage{french}éd.}.
	\item Printing only the author name  before the title, and never the editor, even if the book is anonymous.
 	\item Fixing \biblatex\ bug with  \expandafter\cs\expandafter{\prm{xx}cites} command when using anonymous works.\footnote{\url{https://github.com/plk/biblatex/issues/285}.}
		
\end{enumerate}

\section{Using the package}

\subsection{Loading}
You must load the package after \biblatex package:
\begin{minted}{latex}
\usepackage[…]{biblatex}
\usepackage{biblatex-anonymous}
\end{minted}

\subsection{Sorting with anonymous scheme}

If you want your bibliographies to have all anonymous works at the begining, you can pass option \verb+sorting=anonymous+:

\begin{itemize}
	\item To \biblatex package, to apply it to all bibliographies.
	\item To \cs{printbibliography} command, to apply it to a single bibliography.
\end{itemize}
\subsection{Making exceptions}

If you want to use the \bibfield{editor} or \bibfield{translator} for a entry concerned by this package (e.g. \bibtype{book}, \bibtype{bookinbook} etc.), you can use the \verb+useeditor=true+ or \verb+usetranslator=true+ 
option in the \bibfield{options} field. However, in many case, such entry should be a \bibtype{proceedings} or \bibtype{collection}. 

\subsection{Anonymous work edited on not anonymous books}\label{noinherit}

Sometime, an anonymous \bibtype{bookinbook} is printed in a NOT anonymous work.
If you use cross-referencing system, you would like to NOT inherit the \bibfield{name} from the main entry in the subentry.

\textbf{From \biblatex~3.4.0}, you can use entry option \opt{noinherit=author} as in the following example:

\begin{minted}{latex}
@book{mainentry,
  Author = {Name},
  …
}
@book{subentry,
  Crossref = {mainentry},
  Options = {noinherit=author},
  …
}
\end{minted}
\section{Credits}

This package was created for Maïeul Rouquette's phd dissertation\footnote{\url{http://apocryphes.hypothese.org}.} in 2014. It is licensed on the \emph{\LaTeX\ Project Public License}\footnote{\url{http://latex-project.org/lppl/lppl-1-3c.html}.}. 

Its code was explained before (in French) on Maïeul Rouquette website\footnote{\url{http://geekographie.maieul.net/Gestions-des-oeuvres-anonymes}.}.

All issues can be submitted, in French or English, in the GitHub issues page\footnote{\url{https://github.com/maieul/biblatex-multiple-anonymous/issues}.}.

\section{Change history}

\begin{changelog}

\begin{release}{2.3.0}{2016-09-08}
  \item Add compatibility with \bibtype{bookininbook} and \bibtype{bookininarticle} of \emph{biblatex-bookinbook} package.
\end{release}

\begin{release}{2.2.3}{2016-06-27}
  \item Fix bug added on v.~2.2.2 when using prenote argument.
 \end{release}

\begin{release}{2.2.2a}{2016-06-21}
  \item Fix typo in handbook.
 \end{release}

\begin{release}{2.2.2}{2016-06-18}
  \item Fix \biblatex\ bug with  \expandafter\cs\expandafter{\prm{xx}cites} command when using anonymous works.
  \item Fix bug added by version 2.0.0 when not using the \emph{biblatex-realauthor} package.
 \end{release}
\begin{release}{2.2.1}{2016-06-06}
 \item Fix compatibility with biblatex < 3.4, broken by v. 2.2.0
\end{release}
\begin{release}{2.2.0}{2016-06-05}
  \item Takes account of the \bibfield{realeditor} field of the \emph{biblatex-realauthor} package~2.4.0 and later.
  \item Adds tools to prevent inherinting name from main entry \see{noinherit}.
\end{release}

\begin{release}{2.1.0}{2014-10-30}
\item Coming back, and improve, the \verb+anonymous+ sorting scheme.
\end{release}

\begin{release}{2.0.0}{2014-10-27}
\item Use less complex code, and more \enquote{biblatex friendly} code to suppress editor and translator.
\item Compatibility with \bibtype{bookinincollection} type.
\end{release}

\begin{release}{1.0.0}{2014-07-23}
\item First public release.
\end{release}
\end{changelog}
\end{document}
