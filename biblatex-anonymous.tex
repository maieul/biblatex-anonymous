\documentclass{ltxdockit}[2011/03/25]
\usepackage{btxdockit}
\usepackage{fontspec}
\usepackage[mono=false]{libertine}
\usepackage{microtype}
\usepackage[main=english,french,latin]{babel}
\usepackage[strict]{csquotes}
\setmonofont[Scale=MatchLowercase]{DejaVu Sans Mono}
\usepackage{shortvrb}
\usepackage{pifont}
\usepackage{minted}
% Usefull commands
\newcommand{\biblatex}{biblatex\xspace}
\pretocmd{\bibfield}{\sloppy}{}{}
\pretocmd{\bibtype}{\sloppy}{}{}
\newcommand{\namebibstyle}[1]{\texttt{#1}}
% Meta-datas
\titlepage{%
	title={Managing anonymous work with biblatex},
	subtitle={},
	email={maieul <at> maieul <dot> net},
	author={Maïeul Rouquette},
	revision={2.0.0},
	date={27/10/2014},
	url={https://github.com/maieul/biblatex-anonymous}}

\begin{document}


\printtitlepage
\tableofcontents

\section{Introduction}

In some scholar fields, like history or philology, we have to work with anonymous work. 

This package provides facility to manage them with \biblatex, with citation and bibliography style of the \namebibstyle{verbose} family.

\section{What does the package do?}

This package does those things for  \bibtype{book}, \bibtype{bookinbook}, \bibtype{bookinarticle}, \bibtype{bookinincollection}, \bibtype{mvbook}, \bibtype{inbook}:
\begin{enumerate}
	
	\item Changing label name to author only, never editor.
	\item For French, changing \enquote{\selectlanguage{french}dir.} by \enquote{\selectlanguage{french}éd.}.
	\item Printing only the author name  before the title, and never the editor, even if the book is anonymous.
		
\end{enumerate}

\section{Using the package}

\subsection{Loading}
You must load the package after \biblatex package:
\begin{minted}{latex}
\usepackage[…]{biblatex}
\usepackage{biblatex-anonymous}
\end{minted}

\subsection{Sorting scheme}

With this package, all anonymous works are printed before the works with author. The \verb+anonymous+ sorting scheme is kept only for backward compatibility. 

\subsection{Making exceptions}

If you want to use the \bibfield{editor} or \bibfield{translator} for a entry concerned by this package (e.g. \bibtype{book}, \bibtype{bookinbook} etc.), you can use the \verb+useeditor=true+ or \verb+usetranslator=true+ 
option in the \bibfield{options} field. However, in many case, such entry should be a \bibtype{proceedings} or \bibtype{collection}. 
\section{Credits}

This package was created for Maïeul Rouquette's phd dissertation\footnote{\url{http://apocryphes.hypothese.org}.} in 2014. It is licensed on the \emph{\LaTeX\ Project Public License}\footnote{\url{http://latex-project.org/lppl/lppl-1-3c.html}.}. 

Its code was explained before (in French) on Maïeul Rouquette website\footnote{\url{http://geekographie.maieul.net/Gestion-des-sources-anonymes-avec};\url{http://geekographie.maieul.net/Tri-des-oeuvres-anonymes}.}.

All issues can be submitted, in French or English, in the GitHub issues page\footnote{\url{https://github.com/maieul/biblatex-multiple-anonymous/issues}.}.

\section{Change history}

\begin{changelog}


\begin{release}{2.0.0}{2014-10-27}
\item Use less complex code, and more \enquote{biblatex friendly} code to suppress editor and translator.
\item Suppress \verb+anonymous+ sorting, kept only for backward compatibility.
\item Compatibility with \bibtype{bookinincollection} type.
\end{release}

\begin{release}{1.0.0}{2014-07-23}
\item First public release.
\end{release}
\end{changelog}
\end{document}
