\documentclass{ltxdockit}[2011/03/25]
\usepackage{btxdockit}
\usepackage{fontspec}
\usepackage[mono=false]{libertine}
\usepackage{microtype}
\usepackage[american]{babel}
\usepackage[strict]{csquotes}
\setmonofont[Scale=MatchLowercase]{DejaVu Sans Mono}
\usepackage{shortvrb}
\usepackage{pifont}
\usepackage{minted}
% Usefull commands
\newcommand{\biblatex}{biblatex\xspace}
\pretocmd{\bibfield}{\sloppy}{}{}
\pretocmd{\bibtype}{\sloppy}{}{}
\newcommand{\namebibstyle}[1]{\texttt{#1}}
% Meta-datas
\titlepage{%
	title={Managing anonymous work with biblatex},
	subtitle={},
	email={maieul <at> maieul <dot> net},
	author={Maïeul Rouquette},
	revision={1.0.0},
	date={02/07/2014},
	url={https://github.com/maieul/biblatex-anonymous}}

\begin{document}


\printtitlepage
\tableofcontents


\section{Credits}

This package was created for Maïeul Rouquette's phd dissertation\footnote{\url{http://apocryphes.hypothese.org}.} in 2014. It is licensed on the \emph{\LaTeX\ Project Public License}\footnote{\url{http://latex-project.org/lppl/lppl-1-3c.html}.}. 

Its code was explained before (in French) on Maïeul Rouquette website\footnote{\url{http://geekographie.maieul.net/Gestion-des-sources-anonymes-avec};\url{http://geekographie.maieul.net/Tri-des-oeuvres-anonymes}.}.

All issues can be submitted, in French or English, in the GitHub issues page\footnote{\url{https://github.com/maieul/biblatex-multiple-anonymous/issues}.}.

\section{Change history}

\begin{changelog}



\begin{release}{1.0.0}{2014-07-023}
\item First public release.
\end{release}
\end{changelog}
\end{document}
